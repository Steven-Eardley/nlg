\documentclass[a4paper,11pt,oneside]{article}

\begin{document}

\title{NLG Assignment 2}
\author{Steven Eardley s0934142}
%\maketitle

\section{Question 1.1}
The content plan contains the information that there should be three comparisons between the subjects, along with the corresponding items; the text plans 'flesh out' these structures, with different presentations of the information. The first interleaves the sets of properties, switching subjects. The second groups the properties for each subject, while the third swaps the subjects around: Giovanni's first, Ti Ammo second.

\section{Question 1.2}
No, the rankings do not appear correct. The model tends to prefer shorter sentences, meaning those in simple language which to me sound stilted are promoted to the top of the list. In addition, the top ranked sentence has a problem with adjective order: it says \emph[Italian mediocre] rather than the preferred \emph[mediocre Italian]. Finally, in the top ranked sentences only decor is mentioned, not quality of food, so these sentences don't contain as much information as later ones. These shortcomings may be due to the model being used out of domain - a newswire trained model will have been trained on texts in which brevity is valued, whereas here we wish to be more discriptive.

The top ranked sentence is:
\begin{verbatim}
Giovanni's is a good Italian restaurant, which has plain decor. 
Ti Amo is an Italian mediocre restaurant, which has tacky decor. 
\end{verbatim}

In the MATCH system, a decor appears at the bottom of the list of objectives, whereas food quality is found at the top. We'd prefer this system to also emphasise this information.

\section{Question 1.3}
Yes, there is some improvement. The sentences at the top of the list with incorrect adjective order have gone, only one instance exists far down the list, plus all sentences contain information on food as well as decor.  Another behaviour I attempted to discourage by removal from the corpus was the use of the same comparison phrase repeatedly - it didn't read very well when 'on the other hand' is used twice, for example:

\begin{verbatim}
    Giovanni's and Ti Amo are Italian restaurants. Ti Amo serves mediocre food. Giovanni's, 
    on the other hand, serves good food. Ti Amo has tacky decor. Giovanni's, on the other hand, has plain decor.
\end{verbatim}

Using the second language model did not affect this, however - there were still plenty of examples generated showing this behaviour, including the top two suggestions. This is a restriction with the model: there is no way using n-grams to perform long-reaching word preferences, since they only look at local frequency counts.

It is also clear that the shorter sentences are still preferred, at the cost of some sentence fluency.

\section{Question 1.4}
The web corpus trained language model does not provide an improvement over the WSJ language model, and the problem of repetition is still present. This problem and the issue with preferring shorter strings are unlikely to go away, as stated earlier. Interestingly, when this corpus is used in 50/50, 20/80 or 80/20 conjunction with the WSJ corpus, the results are the same as using the WSJ alone, with the same problem with missing information - only decor is mentioned, not food. For this reason it seems the \verb+restaurants-2+ language model is less useful than its predecessor.

Using only the web sourced model gives slightly better rankings than the WSJ alone - the adjective order is correct in the top ranked one:

\begin{verbatim}
Ti Amo is a mediocre Italian restaurant, which has tacky decor.
Giovanni's is a good Italian restaurant, which has plain decor.
\end{verbatim}

Like the WSJ, we have a loss of information. Using the two restaurant corpora together weighted 50/50 produce results similar to our good results from \verb+restaurants-1+ and the WSJ corpora. This seems to reinforce that \verb+restaurants-1+ is the best we have produced for the moment.

\section{Question 1.5}

Using the \verb+restaurants-3+ language model 50/50 with WSJ produces results indistinguishable from the smaller web corpus. The same errors are present, and the shorter strings which only discuss decor are preferred, once again. It seems there is little benefit to be found by adding more data which is in the same vein as previous useless data. When discarding the WSJ language model the results from \verb+restaurants+ alone look quite similar to the WSJ, but without the adjective order errors in the top ranked few.

The best results so far are found by using all three restaurant models, with comparable weights (34/33/33 respectively since they must sum to 100). The top ranked phrase:

\begin{verbatim}
Ti Amo and Giovanni's are Italian restaurants. Ti Amo serves mediocre food but Giovanni's serves good food. 
Ti Amo has tacky decor but Giovanni's has plain decor. 
\end{verbatim}

This is quite good because it keeps the subjects in the same order each time - it first addresses Ti Amo, then Giovanni's. It groups the similarities in the first sentence then contrasts those that are different.

\section{Question 1.6}

Each language modeal was tested individually, then with different combinations, followed by different weights:
    
\begin{tabular}{ | c | c | c | c | }
  \hline                        
  Run no. & WSJ & r-1 & r-3 \\ \hline
  1 & 100 & 0 & 0 \\
  2 & 0 & 100 & 0 \\
  3 & 0 & 0 & 100 \\
  4 & 50 & 50 & 0 \\
  5 & 50 & 0 & 50 \\
  6 & 0 & 50 & 50 \\
  7 & 0 & 80 & 20 \\
  8 & 0 & 20 & 80 \\
  9 & 33 & 34 & 33 \\
  \hline  
\end{tabular}

The results were all very similar. Using the WSJ LM alone once again led to adjectives being incorrectly ordered near the top (although not the top ranked one). Using the \verb+restaurant-3+ language model did improve on this. The \verb+restaurant-1+ model in this case failed to produce ranks - each sentence was given 0. This is perhaps because it is too specific to the first task, producing comparison strings. Pairing it 50/50 with the WSJ gave the worst output, with errors in the top result. All other blends, on the other hand, produced equivalent results with acceptable outcomes. 

\section{Quesion 1.7}
The probablilities for longer sentences are diminished because the overall sentence production probability is a product of the n-gram frequency probabilities. Since all probabilities are always less than one, the overall probability will always decrease with length as more words are added and their probablities multiplied on. A solution to this is to apply a weight vector to counteract this effect: giving larger weights for each word added. This should be designed to counteract the loss of probability and not to give a preference for vastly long sentences.

\section{Quesion 1.8}
These phrases show the adjective ordering phenomenon as mentioned (many times) above. It shows those I have termed undesirable are found to have lower counts from Google. Using these frequencies, the results could be re-ranked according to how often the contained word orderings are used on the Web. One would find an improvement over the current systems in such a case.

\section{Quesion 2.1}

For text 1:
\begin{verbatim}

\end{verbatim}

For text 2:
\begin{verbatim}

\end{verbatim}

The first text may be preferred because it does a better job of grouping the shared information, namely that they are 'from a strong company'. The second sentence better serves as a justification of the assertion that those two shows are the best, rather than simply describing the shows individually.

\section{Quesion 2.2}
User X seems to prefer short sentences with little information in each, while user Y enjoys long sentences and commas to separate topics. To satisfy them both, don't go to either extreme, rather generate sentences of medium length with only one or two topics and commas.

\section{Quesion 2.3}
My attribute list, in order of importance:
\begin{enumerate}
\item Genre
\item Show's critic score
\item Social network friends going
\item Ticket price
\item Time of performance
\item Proximity to user
\end{enumerate}

Within a real-time system a user (say, someone with limited mobility) may require a different weighting to these preferences. In such a case, proximity to the user and time may feature highly compared to ticket price or the critic score, in order to assure that they can reach it in time. Perhaps a socialite may place having their Facebook friends going to the same show at the top of the list, or even give it a 50\% ranking with the critic score (after all, they'll want to go to the \emph{good} things their friends are at) and discard the remaining options.

MATCH's questionnaire method to assign weights (asking 'what would be the first thing you change'?') would be a good method for choosing weights for the attributes. It may also be possible for a more direct approach which lets the user have full control - presenting an interactive list where the user can move their preference to the top. An alternative and automatic method may be for the program to guess the weights and predict the outcome for a set of possible choices, then adjust the model weights according to the chosen option. This rather depends if the user is willing to put up with a few poor choices while the system is experimenting.


\section{Quesion 2.4}

\end{document}
